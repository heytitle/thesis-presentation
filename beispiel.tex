\documentclass[Nike]{tuberlinbeamer}

%\usepackage{arev} % Empfehlenswert, wenn Formeln gesetzt werden


\title{Das ist ein sehr langer Titel für meinen überaus wichtigen Vortrag}
\subtitle{Untertitel}
\author{Vorname Nachname}
\institute{Technische Universität Berlin}


%% Nur für dieses Dokument
\usepackage{listings}
\lstset{basicstyle=\small\ttfamily,tabsize=2,numbers=left,numberstyle=\tiny\color{gray},rulecolor=\color{gray},frame=single,framerule=0.7pt,frameround=tttt,xleftmargin=5mm,xrightmargin=2mm,language=[LaTeX]TeX}
\newcommand{\tubcls}{\texttt{tuberlinbeamer}\xspace}
%%


\begin{document}


\begin{frame}
\maketitle
\end{frame}

\begin{frame}
\tableofcontents
\end{frame}


\section{Einleitung}

\begin{frame}[fragile]{Verwendung der \tubcls-Klasse}
Es folgen demnächst ein paar Folien zur Verwendung dieser Dokumentklasse.
\begin{itemize}
\item Kenntnis der \emph{beamer}-Klasse ist von Vorteil
\end{itemize}
\end{frame}


\section{Zusammenfassung}

\begin{frame}{ToDo}
\begin{itemize}[<+->]
\item \emph{ToDo} schreiben
\item \emph{ToDo} abarbeiten
\end{itemize}
\end{frame}


\end{document}
